\documentclass[a4paper,11pt]{article}
\usepackage{color} 
\renewcommand{\baselinestretch}{1.2}% Comando para incrementar el espacio % 1.0
                                    % entre las lineas: simple 1 (default)
                                    %                   renglon y medio 1.5
                                    %                   doble espacio 2

\setlength{\parskip}{3pt}           % distancia entre paragrafos

 % Own new commnads
\newcommand \RR{\hbox{I\kern -.2em \hbox {R}}}
\newcommand \al{^}

\newcommand{\Rset}{{\mathbb R}}  % Set of real numbers
\newcommand{\Cset}{{\mathbb C}}  % Set of complex numbers
\newcommand{\Nset}{{\mathbb N}}  % Set of complex numbers
\newcommand{\Pset}{{\mathbb P}}  % Set of complex numbers
\definecolor{RoyalBlue}{rgb}{0.25,0.41,0.88}



\textwidth 15.5cm
\topmargin -2cm
%\parindent 0cm  %Espacio en la primera letra al inicio de un parrafo
\textheight 24cm
\parskip 1mm

\begin{document}

\title{Manual para la Clase Polinomio}

\date{}
\maketitle

\noindent
\textbf{Descripci\'on}

El objetivo de la plantilla de la clase \textit{Polinomio} -codificada en C++- es la creaci\'on, 
evaluaci\'on y derivaci\'on de polinomios ($p \in P_{m-1}(\RR^d)$) requeridos en 
la interpolaci\'on con funciones de base radial y en la soluci\'on num\'erica de ecuaciones en 
derivadas parciales. Al ser una plantilla (template) podemos instanciar el objeto Polinomio
con distintos tipos de datos, por ejemplo: float, double, o alg\'un tipo de dato definido por
el usuario. Para evitar conflictos con los nombres de las clases, definimos el espacio de
nombre \textit{pol}.

Las caracter\'istica principales de la clase \textit{Polinomio} son:
\begin{itemize}
  \item Datos en $\RR^d$. !!Funciona para datos en cualquier dimensi\'on, esta es la diferencia
        principal con las versiones anteriores.
  \item Crea polinomios de cualquier grado.
  \item C\'alculo de las derivadas de cualquier orden y sobre cualquier variable.
  \item Operaciones binarias entre polinomios: suma, resta y multiplicaci\'on por un escalar.
\end{itemize}

\vspace{3mm}
\noindent
\textbf{C\'odigo Fuente}: polinomio\_v\_1.9.cpp

\noindent
\textbf{Desarrollado}: Jos\'e Antonio Mu{\~n}oz G\'omez

\noindent
\textbf{Sugerencias}: jose.munoz@cucsur.udg.mx

\vspace{3mm}
\footnotesize{
El presente manual forma parte del proyecto "M\'etodos de Funciones Radiales para la Soluci\'on
de Ecuaciones en Derivadas Parciales" www.dci.dgsca.unam.mx/pderbf, coordinado por  P\'edro
Gonz\'alez-Casanova.}


\newpage
\vspace{3mm}
\noindent
\begin{center}
\begin{Large}
Funciones de la Clase Polinomio
\end{Large}
\end{center}
%-----------------------------------------
En la carpeta de \textit{examples} se muestran 22 ejemplos documentos de como emplear
la clase Polinomio. La declaraci\'on de un polinomio se realiza mediante la instrucci\'on
\begin{center}
 Polinomio$<$T$>$ p;
\end{center}
donde T es un tipo de dato en C o en C++, por ejemplo T puede ser un float o double.
Se recomienda utilizar
\begin{center}
 Polinomio$<$double$>$ p;
\end{center}
En lo subsecuente se emplear\'a a T para indicar el tipo de dato empleado.

%-----------------------------------------
\vspace{4mm}
\noindent
{\textcolor{RoyalBlue}{ \textbf{make(d, m)} }

\noindent
Descripci\'on: crea internamente un polinomio $p \in P^d_{m-1}$, el polinomio internamete se almacena como
$p=\{p_1, p_2, \ldots, p_M\}$. Posteriormente se puede evaluar u obtener las derivadas de $p$.

\noindent
Par\'ametros de Entrada

\begin{tabular}{ll}
    d  &  ENTERO que especifica la dimensi\'on del espacio. \\

   m   &  ENTERO que especifica el orden $m$ del polinomio.
\end{tabular}


%-----------------------------------------
\vspace{7mm}
\noindent
{\textcolor{RoyalBlue}{ \textbf{deriva(parcial, orden)} }


\noindent
 Descripci\'on: deriva el polinomio $p(x)$ creado con \textit{make}.

\noindent
Par\'ametros de Entrada


\begin{tabular}{ll}
 parcial  &  ENTERO vector con los datos, longitud $d \cdot n$. \\
          &  STRING vector con los datos, longitud $d \cdot n$. \\
  orden   &  ENTERO que especifica el orden de derivaci\'on.
\end{tabular}


\vspace{5mm}
\noindent
Ejemplos:

Para obtener la primera derivada del polinomio con respecto de la variable $x$ para datos en una dimensi\'on se puede realizar:

\begin{center}
deriva("x",1) \hspace{1cm} o  \hspace{1cm} deriva(1,1)
\end{center}

Para obtener la segunda derivada del polinomio con respecto de la variable $y$ para datos en dos dimensiones $\bar{x} = (x,y)$
se puede realizar:

\begin{center}
deriva("y",1) \hspace{1cm} o  \hspace{1cm} deriva(2,1)
\end{center}

Para obtener la primera derivada del polinomio con respecto de la variables $z$ para datos en tres dimensiones $\bar{x} = (x,y,z)$
se puede realizar:

\begin{center}
deriva("z",1) \hspace{1cm} o \hspace{1cm}  deriva(3,1)
\end{center}

Para datos definidos en $d>3$, se debe utilizar como primer argumento un entero.


%-----------------------------------------
\vspace{7mm}
\noindent
{\textcolor{RoyalBlue}{ \textbf{eval(x,  d,  px, dim-px)} }

\noindent
Descripci\'on: eval\'ua cada elemento de la base del polinomio creado en un punto $x \in \RR^d$.
Recordando que $p=\{p_1, p_2, \ldots, p_M\}$, entonces la evaluaci\'on de $x$ en $p$ corresponde a
$p(x)=\{p_1(x), p_2(x), \ldots, p_M(x)\}$.
Si el polinomio fue previamente derivado entonces evalua la derivada del polinomio en dicho punto. 

\noindent
Par\'ametros de Entrada

\begin{tabular}{ll}
    x   &  T* vector con un elemento $x \in \RR^d$, $x$ tiene longitud $d$. \\
    d   &  ENTERO que especifica la dimensi\'on del espacio. \\
dim-px  &  ENTERO que especifica el n\'umero de elementos del polinomio. 
\end{tabular}

\noindent
Par\'ametros de Salida

\begin{tabular}{ll}
   px \hspace{7mm}  &  T* vector de longitud dim-px.
\end{tabular}


\vspace{5mm}
\noindent
La longitud de px depende de los par\'ametros $m$, $d$ utilizados en la funci\'on \textit{make}.
Para conocer el valor de dim-px se requiere utilizar la funci\'on \textit{get\_M}, la cual
regresa el n\'umero de elementos que tiene el polinomio creado.

%-----------------------------------------
\vspace{7mm}
\noindent
{\textcolor{RoyalBlue}{ \textbf{ P $\leftarrow$ build(x, n, d)  } }

\noindent
Descripci\'on: crea una matriz $P$ de dimensiones $n \times M$ la cual contiene por rengl\'on
la evaluaci\'on del polinomio $p$ en un punto $x$ (ver funci\'on \textit{eval}).
Si el polinomio fue previamente derivado entonces la matriz $P$ contiene la evaluaci\'on del
polinomio derivado evaluado en dichos puntos.

\noindent
Par\'ametros de Entrada

\begin{tabular}{ll}
    x  &  DOUBLE vector con los datos, longitud $d*n$. \\
    n  &  ENTERO que especifica el n\'umero de elementos del vecto $x$. \\
    d  &  ENTERO que especifica la dimensi\'on del espacio.
\end{tabular}


\noindent
Par\'ametros de Salida

\begin{tabular}{ll}
    P  &  T** matriz de dimensiones $n \times M$.
\end{tabular}

\noindent
La funci\'on \textit{build} es requerida para construir el polinomio en la matriz de Gramm, as\'i como
la submatriz -corresponde al polinomio derivado- requerida en el esquema de colocaci\'on asim\'etrico de E. J. Kansa.


%-----------------------------------------
\vspace{7mm}
\noindent
{\textcolor{RoyalBlue}{ \textbf{ build(x, n, d, P, k, m, op)  } }
	
\noindent
Descripci\'on: evalua el polinomio $p$ en los puntos $x$, el resultado lo
almacena en un vector de datos $P$.
Si el polinomio fue previamente derivado entonces  $P$ contendra la evaluaci\'on del
polinomio derivado evaluado en dichos puntos.
En esta funci\'on no se reserva la memoria para $P$, esta debe ser previamente creada. 

\noindent
Par\'ametros de Entrada

\begin{tabular}{ll}
    x  &  T* vector con los datos, longitud $d \cdot n$. \\
    n  &  ENTERO que especifica el n\'umero de elementos del vecto $x$. \\
    d  &  ENTERO que especifica la dimensi\'on del espacio. \\
    k  &  ENTERO que especifica el n\'umero de renglones de $P$, k debe coincidir con n.\\ 
    m  &  ENTERO que especifica el n\'umero de columnas de $P$, m debe coincidir  con dim-px.  \\
    op &  ENTERO si op=0 se almacenan los datos en orden ROW-MAJOR, si op=1  \\    
       &   se almacenan los datos en orden COL-MAJOR como en Fortran. Si no se \\
       &  especifica el par\'ametro op, este vale 0.
\end{tabular}

\noindent
Par\'ametros de Salida

\begin{tabular}{ll}
    P  &  T*  vector de longitud $n \cdot m$. Requiere que previamente se reserve la memoria para P.
\end{tabular}

\noindent
La funci\'on \textit{build} es requerida para construir el polinomio en la matriz de Gramm, as\'i como
la submatriz -corresponde al polinomio derivado- requerida en el esquema de colocaci\'on asim\'etrico de E. J. Kansa.


%----------------------------------------- 
\vspace{7mm}
\noindent
{\textcolor{RoyalBlue}{ \textbf{ Matrix$<$T$>$ $\leftarrow$ build$\_$tnt(x, n, d)  } }

\noindent
Descripci\'on: cuando se compila con la libreria Template Numerical ToolKit, ver README,
se crea una matriz $P$ de dimensiones $n \times M$ la cual contiene por rengl\'on
la evaluaci\'on del polinomio $p$ en un punto $x$ (ver funci\'on \textit{eval}).
Si el polinomio fue previamente derivado entonces la matriz $P$ contiene la evaluaci\'on del
polinomio derivado evaluado en dichos puntos. Los par\'ametros de entrada son similares
a la primera funci\'on \textit{build}.

%-----------------------------------------
\vspace{7mm}
\noindent
{\textcolor{RoyalBlue}{ \textbf{+, -} }

\noindent
Descripci\'on: sean $p,q  \in P_{m-1}(\RR^d)$,  la clase Polinomio tiene m\'etodos p\'ublicos
para poder realizar las siguientes tres operaciones:

\[
  p + q, \hspace{7mm}
  p - q, \hspace{7mm}
 \alpha \cdot p
\]
con $\alpha \in \RR$. Las tres operaciones anteriores son requeridas cuando trabajamos con ecuaciones
en derivadas parciales.


%-----------------------------------------
\vspace{7mm}
\noindent
{\textcolor{RoyalBlue}{\textbf{ int get$\_$m()  } }

\noindent
Descripci\'on: obtiene el grado del polinomio $m$ utilizada
en la construcci\'on del polinomio con la funci\'on \textit{make}.

%-----------------------------------------
\vspace{7mm}
\noindent
{\textcolor{RoyalBlue}{\textbf{ int get$\_$d()  } }

\noindent
Descripci\'on: obtiene la dimensi\'on del espacio $d$ utilizada
en la construcci\'on del polinomio con la funci\'on \textit{make}.

%-----------------------------------------
\vspace{7mm}
\noindent
{\textcolor{RoyalBlue}{\textbf{ int get$\_$M()  } }

\noindent
Descripci\'on: obtiene el n\'umero de elementos $M$ que conforman
la base del polinomio: $p(x)=\{p_1(x), p_2(x), \ldots, p_M(x)\}$.
El valor de $M$ se determina internamente en la funci\'on \textit{make} y
es calculado como:
\[
M=
\left(
\begin{array}
[c]{c}%
d+m-1\\
m-1
\end{array}
\right)
\]



%-----------------------------------------
\vspace{7mm}
\noindent
{\textcolor{RoyalBlue}{ \textbf{string version()  } }

\noindent
Descripci\'on: obtiene la versi\'on de la clase Polinomio.

%-----------------------------------------

\newpage
\vspace{3mm}
\noindent
\begin{center}
\begin{Large}
Ejemplos
\end{Large}
\end{center}

El siguiente ejemplo en 1d contenido en /examples/1d/eje$\_$9.cpp muestra las distintas 
formas de construir la matriz P. Para compilar el programa s\'olo realize: c++ $\mbox{eje}\_\mbox{9.cpp}$,
el archivo compilado ser\'a a.out, el cual arroja de salida:
\begin{verbatim}
Calculo de la matriz P con   p(x) = 1 + x + x*x

P = 
    1.0 1.1 1.2 
    1.0 1.2 1.4 
    1.0 1.3 1.7 

P = 
    1.0  1.1  1.2  1.0  1.2  1.4  1.0  1.3  1.7  

P = 
    1.0  1.0  1.0  1.1  1.2  1.3  1.2  1.4  1.7 
\end{verbatim}
Para esclarecer detalles, se muestra el programa utilizado.    
\begin{verbatim}
int main(void)
{ 
 Polinomio<double> p;
 int       n,m,d;
 double    x[3];
 double    px[2];
 int       dim_px;
 double    **P1;
 double    *P2;
 double    *P3;


//muestro algunos datos
  cout<<endl;
  cout<<"Calculo de la matriz P con   p(x) = 1 + x + x*x"<<endl;
  cout<<endl;

//defino la dimension del espacio (datos en una dimension) y orden del polinomio
  d=1; m=3;

//construyo el polinomio p(x) = 1 + x + x*x + x*x*x
  p.make(d,m); 

//obtengo el numero de elmentos del polinomio
  dim_px = p.get_M();

//defino los puntos a evaluar
   n = 3;
   x[0] =  1.1;  x[1] =  1.2;  x[2] =  1.3;

//construyo la matriz con las evaluaciones del polinomio en los nodos
     P1 = p.build( x, n, d);

   //muestro la matriz
     show_mat(P1,  n, dim_px);
     cout<<endl;
	
//construyo la matriz con las evaluaciones del polinomio en los nodos
//empleo ROW_MAJOR	
	 P2 = new double[n*dim_px];
	 p.build( x, n, d, P2, n, dim_px);
	 
	//muestro la matriz
	  show_mat(P2,n,dim_px);
	  cout<<endl;	

//construyo la matriz con las evaluaciones del polinomio en los nodos
//empleo COL_MAJOR	
	 P3 = new double[n*dim_px];
	 p.build( x, n, d, P3,n,dim_px,1);
	
	//muestro la matriz
	 show_mat(P3,n,dim_px);
	 cout<<endl;	

 return 0;
}

\end{verbatim}
%-----------------------------------------

\end{document}

